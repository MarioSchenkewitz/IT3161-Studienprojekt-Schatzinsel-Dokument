%% Vorlage Bachelorarbeit

%% Versionshistorie:

%% v1.0: Erstellung durch Johannes Woske, IT2010, j.woske+latex@gmail.com
%% v2.0: Überarbeitung und Ergänzung durch Anne Traulsen, IT2015, a.traulsen+latex@gmail.com

\documentclass[
	12pt, %Schriftgröße
	a4paper,
	listof=totoc, %Inhaltsverzeichniseinträge für Listen (z.B. Abbildungen)
	bibliography=totoc, %Inhaltsverzeichniseinträge f+r Quellen
	numbers=noenddot, %Entfernt Punkt hinter Gliederungsnummern
	ngerman, %Sprachpaket
	headsepline, %Headertrennlinie
	%footsepline, %Footertrennlinie
	oneside %einseitiges Druckformat %%% Unterdrücken der leeren Seite nach Titelblatt
	]{scrbook} %Dokumentenklasse (Koma-Script)
\usepackage[section]{placeins}
\usepackage[T1]{fontenc}
\usepackage{float}
\usepackage[utf8]{inputenc}
\usepackage[ngerman]{babel}
\usepackage{url}
\usepackage{graphicx} %Bilder einfügen
%\usepackage{pdfpages} %PDF einfügen
\usepackage[a4paper, margin=1in]{geometry}
\usepackage[right]{eurosym} %Euro-Zeichen
\usepackage{amssymb}
\usepackage{cite} %Quellenangaben
\usepackage{setspace} % Zeilenabstand
\usepackage[ 
   colorlinks,        % Links ohne Umrandungen in zu wählender Farbe 
   linkcolor=black,   % Farbe interner Verweise 
   filecolor=black,   % Farbe externer Verweise 
   citecolor=black,   % Farbe von Zitaten 
   urlcolor=blue	  % Farbe von Links
   ]{hyperref} %Verlinkungen
\usepackage{bookmark}
\usepackage[figure]{hypcap}
\usepackage[ngerman]{translator}
\usepackage{blindtext} % Lorem-Ipsum-Plugin
\usepackage{scrhack}
%\usepackage[
	%nonumberlist, %keine Seitenzahlen anzeigen
	%acronym,      %ein Abkürzungsverzeichnis erstellen
	%toc,          %Einträge im Inhaltsverzeichnis
	%section      %im Inhaltsverzeichnis auf section-Ebene erscheinen
	%]
%{glossaries}
\usepackage{outlines}

\usepackage{listings,xcolor} %Codeanzeige
\usepackage[normalem]{ulem}
\useunder{\uline}{\ul}{}


\usepackage{mathtools}
\DeclarePairedDelimiter\abs{\lvert}{\rvert}%
\DeclarePairedDelimiter\norm{\lVert}{\rVert}%
% Swap the definition of \abs* and \norm*, so that \abs
% and \norm resizes the size of the brackets, and the 
% starred version does not.
\makeatletter
\let\oldabs\abs
\def\abs{\@ifstar{\oldabs}{\oldabs*}}
%
\let\oldnorm\norm
\def\norm{\@ifstar{\oldnorm}{\oldnorm*}}
\makeatother

\usepackage{chngcntr}
\counterwithout{figure}{chapter}
\counterwithout{table}{chapter}



%%%%%%%%%%%%%%%%%%%%%%%%%
%%
%% Code Definition
%%
\lstdefinelanguage{powershell}%
  {
	alsodigit = {-},   
   morekeywords={abstract,break,case,catch,const,continue,do,else,elseif,%
      end,export,false,for,function,immutable,import,importall,if,in,%
      macro,module,otherwise,quote,return,switch,true,try,type,typealias,%
      using,while,new-object,psobject,Get-Command,Process,Get-ADUser,Select,ActiveXObject},%
   sensitive=true,%
   alsoother={\$},%
   morecomment=[l]\#,%
   morecomment=[n]{\#=}{=\#},%
   morestring=[s]{"}{"},%
   morestring=[m]{'}{'},%
}[keywords,comments,strings]%

%\lstset{
%	language         = [Sharp]C,
%	basicstyle       = \ttfamily,
%    numbers=left, 
%    numberstyle=\tiny, 
%    numbersep=5pt,
%    breaklines=true,
%    frame=single,
%    escapeinside={(*@}{@*)}, %nicht anzuzeigende Ausdrücke, z.B. für Labels
%    language=sh,
%    basicstyle=\ttfamily\fontsize{10}{12}\selectfont,
%    keywordstyle    = \color{dkblue},
%    stringstyle     = \color{red},
%    identifierstyle = \color{black},
%    commentstyle    = \color{gray},
%    emph            =[1]{php},
%    emphstyle       =[1]\color{black},
%    emph            =[2]{if,and,or,else,Parameter,-Property},
%    emphstyle       =[2]\color{dkyellow},
%    } 

\usepackage{color}
\definecolor{bluekeywords}{rgb}{0,0,1}
\definecolor{greencomments}{rgb}{0,0.5,0}
\definecolor{redstrings}{rgb}{0.64,0.08,0.08}
\definecolor{xmlcomments}{rgb}{0.5,0.5,0.5}
\definecolor{types}{rgb}{0.17,0.57,0.68}

\usepackage{listings}

\lstset{language=csh,
captionpos=b,
%numbers=left, %Nummerierung
%numberstyle=\tiny, % kleine Zeilennummern
frame=single, % Kasten, lines alternativ
showspaces=false,
showtabs=false,
breaklines=true,
showstringspaces=false,
breakatwhitespace=true,
escapeinside={(*@}{@*)},
commentstyle=\color{greencomments},
morekeywords={partial, var, value, get, set},
keywordstyle=\color{bluekeywords},
stringstyle=\color{redstrings},
basicstyle=\ttfamily\small,
captionpos=b,
numbers=left, 
numberstyle=\tiny, 
numbersep=5pt
}

%%%%%%%%%%%%%%%%%%%%%%%%%%%%%%%%%%%%%%%%%%%%%%%%%%%%%
%%%%%%%%%%% Sonderformatierung
%%%%%%%%%%%%%%%%%%%%%%%%%%%%%%%%%%%%%%%%%%%%%%%%%%%%%

% Seitenabstände definieren
\geometry{verbose,tmargin=3cm,bmargin=2cm,lmargin=3cm,rmargin=3cm} 

% Hurenkinder und Schusterjungen verhindern (Ja, das heißt wirklich so!!!)
\clubpenalty = 10000 \widowpenalty = 10000 \displaywidowpenalty = 10000 

\newcommand{\footfigref}[1]{\footnote{Abb. \ref{#1} auf Seite \pageref{#1}}}

%% Bei Referenzen im Text wird jetzt bei allen Ebenen "Kapitel" vorgestellt, z.b. Kapitel 2, Kapitel 2.2, Kapitel 6.3.2
\addto\extrasngerman{%
    \def\sectionautorefname{Kapitel}%
    \def\subsectionautorefname{Kapitel}%
    \def\subsubsectionautorefname{Kapitel}%
    }

% Vertikaler Abstand zwischen Ende Textblock - Ende Fußzeile --> Abstand der Seitenzahl von Rand erhöhen 
\setlength{\footskip}{10mm}

% Abstand vor/nach Überschriften verändern

\RedeclareSectionCommand[%
    beforeskip=0.5\baselineskip,
    afterskip=0.5\baselineskip
]{chapter}

\RedeclareSectionCommand[%
    beforeskip=0.5\baselineskip,
    afterskip=0.5\baselineskip
]{section}

\RedeclareSectionCommand[%
    beforeskip=0.1\baselineskip,
    afterskip=0.1\baselineskip
]{subsection}

\RedeclareSectionCommand[%
    beforeskip=0.01\baselineskip,
    %%afterskip=0.2\baselineskip
]{paragraph}

\usepackage[acronym, nonumberlist]{glossaries} %% use after hyperref %Glossar-Paket laden


\setlength{\abovecaptionskip}{4pt}  % 1pc=12pt 
\setlength{\belowcaptionskip}{0pt}
%\setlength{\textfloatsep}{4pt}
\setlength{\intextsep}{1pc}

%% Verkleinerung der Textgröße unter Abbildungen
\addtokomafont{caption}{\small}

% falsche Default-Silbentrennung überschreiben
\include{hyphenation}

% Den Punkt am Ende der Glossareinträge deaktivieren
\renewcommand*{\glspostdescription}{}

%Glossar-Befehle anschalten
%\makeglossaries
\makenoidxglossaries

% sorgt dafür, dass bei Leerzeile die Einrückung verhindert und stattdessen eine Leerzeile eingefügt wird % erspart bigskips und erhöht die Lesbarkeit im LaTeX-Text 
\KOMAoptions{parskip=full*}

% ändert Titelschriftart in Serifen-Normalschriftart
\addtokomafont{disposition}{\rmfamily} 



\loadglsentries{glossar.tex}

%%%%%%%%%%%%%%%%%%%%%%%%%%%%%%%%%%%%%%%%%%%%%%%%%%%%%
%%%%%%%%%%% Textbausteine
%%%%%%%%%%%%%%%%%%%%%%%%%%%%%%%%%%%%%%%%%%%%%%%%%%%%%
%%%%%%%%%%%% Studentennamen
\newcommand{\studentNameEins}{Jahn, Marko}
\newcommand{\studentNameZwei}{Schenkewitz, Mario}
\newcommand{\studentNameDrei}{Zilius, Sven}
%%%%%%%%%%%% Typ der Arbeit
\newcommand{\type}{Studienprojekt I}
%%%%%%%%%%%% Thema
\newcommand{\topic}{Schatzinsel}
%%%%%%%%%%%% Untertitel
\newcommand{\subtopic}{Entwickeln eines Spiels durch objektorientiertes Programmieren}
%%%%%%%%%%%% Studienbereich
\newcommand{\studienbereich}{Technik}
%%%%%%%%%%%% Fachrichtung
\newcommand{\fachrichtung}{Duales Studium Wirtschaft • Technik}
%%%%%%%%%%%% Studiengang
\newcommand{\studiengang}{Informatik}
%%%%%%%%%%%% Betrieb
\newcommand{\company}{}
%%%%%%%%%%%% Betreuer HWR
\newcommand{\betreuerHS}{Zimmermann, Arthur}
%%%%%%%%%%%% Betreuer Unternehmen
\newcommand{\betreuerUnt}{}
%%%%%%%%%%%% Jahrgang
\newcommand{\jahrgang}{2020}
%%%%%%%%%%%% Matrikelnummer
\newcommand{\MatrEins}{}
\newcommand{\MatrZwei}{685593}
\newcommand{\MatrDrei}{}
%%%%%%%%%%%% Modul
\newcommand{\Modul}{IT3161, Studienprojekt I}
%%%%%%%%%%%% Anzahl Wörter
\newcommand{\wordsnr}{TDB}
%%%%%%%%%%%% Fertigstellungsdatum
\newcommand{\Dateofcompletion}{TBD}
%%%%%%%%%%%%%%%%%%%%%%%%%%%%%%%%%%%%%%%%%%%%%%%%%%%%%>>>>>>>


\begin{document}

%%%%%%%%%%%%%%%%%%%%%%%%%%%%%%%%%%%%%%%%%%%%%%%%%%%%%>>>>>>>
%%%%%%%%%%% Titelblatt

%% Anordnung und Aussehen von Titel und Untertitel

\subject{\type}

\title{
\normalfont\endgraf\rule{\textwidth}{.4pt}
\begingroup
	\centering
	\linespread{1.5}
	\huge\topic
\endgroup
\linespread{1}
\ \\ % Falls kein Subtopic, auskommentieren
\ \\ % Falls kein Subtopic, auskommentieren
\large\subtopic % Falls kein Subtopic, auskommentieren
\endgraf\rule{\textwidth}{.4pt}
}
 
%%Eigentlich nicht besonders schön, aber Koma erlaubt die Anordnung eines weiteren Felden (hier: Fachbereich) nicht
\date{\normalsize fertiggestellt am \Dateofcompletion \\ \textbullet \\ Fachbereich Duales Studium Wirtschaft / Technik \\
Hochschule für Wirtschaft und Recht Berlin}
%% \date muss leer angegeben werden, um die Default-Datumsanzeige zu unterdrücken

\publishers{
	\begin{tabular}{l l}
	\textbf{\normalsize{}} & \normalsize{}  \tabularnewline
	%\textbf{\normalsize{}} & \normalsize{}  \tabularnewline
	\textbf{\normalsize{Name:}} & \normalsize{\studentNameEins, \MatrEins} \tabularnewline
	\textbf{\normalsize{Name:}} & \normalsize{\studentNameZwei, \MatrZwei} \tabularnewline
	\textbf{\normalsize{Name:}} & \normalsize{\studentNameDrei, \MatrDrei} \tabularnewline
	\textbf{\normalsize{Fachrichtung:}} & \normalsize{\fachrichtung} \tabularnewline
	\textbf{\normalsize{Studiengang:}} & \normalsize{\studiengang} \tabularnewline
	\textbf{\normalsize{Studienjahrgang:}} & \normalsize{\jahrgang} \tabularnewline
	\textbf{\normalsize{Betreuer HS:}} & \normalsize{\betreuerHS} \tabularnewline
	\textbf{\normalsize{Anzahl der Wörter:}} & \normalsize{\wordsnr}\\
	
%	\normalsize Vom Betreuer zur Kenntnis genommen:\vspace{2em}\\
%	\end{tabular}
%	\normalsize
%	\begin{tabular}{lp{10em}l} 
% 	\hspace{1cm}   && \hspace{4cm} \\\cline{1-1}\cline{3-3} 
% 	Ort, Datum     && \betreuerHS
	\end{tabular}
	}

\titlehead{\begin{center}
    \includegraphics[scale=1]{bilder/HWR-Logo.jpg}
    \end{center}
    }

\maketitle

\onehalfspacing % anderthalbfacher Zeilenabstand


%%%%%%%%%%%%%%%%%%%%%%%%%%%%%%%%%%%%%%%%%%%%%%%%%%%%%%%%%%%%%%%%%%%%%%%%%%%%%%%%%%%%%%%%%%%%%%%%%%%%%%%%%%%%%%%%%%%%%%%%%%%
%%%%%%%%%%% Dokumenteninhalt START
%%%%%%%%%%%%%%%%%%%%%%%%%%%%%%%%%%%%%%%%%%%%%%%%%%%%%%%%%%%%%%%%%%%%%%%%%%%%%%%%%%%%%%%%%%%%%%%%%%%%%%%%%%%%%%%%%%%%%%%%%%%

%%%%%%%%%%%%%%%%%%%%%%%%%%%%%%%%%%%%%%%%%%%%%%%%%%%%%
%%%%%%%%%%% Abstract
\chapter*{Abstract}
\addcontentsline{toc}{chapter}{Abstract}


\pagenumbering{Roman} % römische Seitenzahlen

\chapter*{Kurzfassung}
\addcontentsline{toc}{chapter}{Kurzfassung}


%%%%%%%%%%%%%%%%%%%%%%%%%%%%%%%%%%%%%%%%%%%%%%%%%%%%%
%%%%%%%%%%% Inhaltsverzeichnis, Tabellen, Abbildungen, etc.
\newpage

\tableofcontents{}
\addcontentsline{toc}{chapter}{Inhaltsverzeichnis}

\listoffigures
%\listoftables

%\section*{Hinweis}
%
%Aus Gründen der besseren Lesbarkeit wird im Text verallgemeinernd die weibliche Form verwendet. Diese Formulierungen umfassen gleichermaßen weibliche und männliche Personen.
%\clearpage

\addcontentsline{toc}{chapter}{Akronyme}
\printnoidxglossaries

\clearpage

%% arabische Seitenzahlen
\pagenumbering{arabic}

%%%%%%%%%%%%%%%%%%%%%%%%%%%%%%%%%%%%%%%%%%%%%%%%%%%%%
%%%%%%%%%%% Einführung

\chapter{Einleitung}\label{sec:einleitung}
Die \gls{OOP} ist ein heutzutage weit verbreitetes Computerprogrammiermodell oder Programmierparadigma. In der \gls{OOP} werden Anwendungen um so genannte Objekte und deren Daten herum strukturiert und nicht um Funktionen oder Logik. So fokussiert sich die \gls{OOP} auf Objekte mit denen eine Anwendung interagiert, diese Objekte können als Datenfeld mit Attributen und eigenem Verhalten angesehen werden.

Einen tieferen Einblick in das Thema der \gls{OOP} soll der Abschnitt \ref{sec:Gundlagen} bieten, in dem die Basis für die \gls{OOP} näher erörtert wird.

Thema dieses Studienprojektes ist es sich dieses Programmierparadigma zu nutze zu machen um ein Spiel zu entwickeln.\\
Der Titel des Spiels ist \glqq Schatzinsel\grqq, es handelt von drei Piraten die sich auf einer Insel befinden, auf der ebenfalls ein Schatz versteckt ist. Es ist ihnen unbekannt wo sich dieser Schatz befinden. Durch zufällige Bewegung über die Insel wollen sie versuchen diesen Schatz zu finden. Jeder der Piraten hat verschiedene Eigenschaften, wie zum Beispiel Geschwindigkeit.\\
Sobald einer der Piraten den Schatz entdeckt, ruft er die anderen zu sich. Am Ende versammeln sie sich alle um den Schatz.

Dieser Umstand soll durch das Spiel modelliert und gegebenenfalls erweitert werden.

\chapter{Grundlagen}\label{sec:Gundlagen}
\section{Objektorientierte Programmierung}
\gls{OOPL} eignen sich durch die Konzentration auf die Objekte mit denen das Programm interagieren soll dazu, dass Software Wartbar und lesbar bleibt. Objekte und ihre Methoden können getrennt von einander erstellt und Programmiert werden. Diese Unterteilung in Einzelteile ermöglicht des weiteren die Wiederverwendbarkeit von Code, also dem Quelltext, entweder in dem die Klasse selbst wiederverwendet wird, oder eine andere Klasse von ihr erbt. Dazu mehr im Abschnitt \ref{sec:Vererbung}

\gls{OOP} ist ein Programmierstil, der mittlerweile in Unternehmen, Industrie und akademischen Kreisen seinen Platz gefunden hat. Es wird ein natürlicheres und dadurch ein mächtigeres Entwurfsparadigma insofern erwartet, dass ein Programmierer produktiver ist, wenn das Anwendungsmodell näher am Problem modelliert werden kann, dass gelöst wird.\\
Anders gesagt ist eine \gls{OOPL} mächtiger, weil es einem Programmierer erlaubt ein Problem zu lösen, dass die Struktur einer Welt nachbildet in welcher das Problem entstand. Vertrautheit mit solch einer Sprache eröffnet Wege zu einer klaren und natürlichen Lösung. \cite{OOPL}

Es werden also im betrachteten Fall der Schatzinsel die Objekte, die aus der echten Welt bekannt sind, so modelliert, dass die Interaktionen und Verhaltensweisen natürlich in einer Anwendung modelliert werden können.

\section{Datenabstraktion und Verkapselung}
Einer der ersten Schritte bei der \gls{OOP} ist es alle Objekte zu sammeln, die manipuliert werden müssen. Im Fall vom Spiel \glqq Schatzinsel\grqq gibt es zum Beispiel Piraten, eine Insel, ein Schatz, aber auch eine graphische Benutzeroberfläche die alle als Objekt betrachtet werden können. Sie müssen untereinander aber auch mit dem Benutzer interagieren können.

Einer der großen Durchbrüche bei der Gestaltung von Programmiersprachen geschah, als es ermöglicht wurde, dass Programmierer eigene Datentypen definieren konnten. C.A.R. Hoare's record class in Simula67 gilt als die Quelle der Idee.\cite{OOPL}
Verbreitung der durch Programmierer definierten fand durch die Sprache Pascal statt. \cite{Pascal}

So kann in Pascal ein Verbund definiert werden, der selbst aus Daten besteht. 

\begin{lstlisting}[language=Pascal, caption=Verbund Beispiel Definition \cite{PascalVerbund}]
TYPE Auto = RECORD
KFZ: String[12];
Gewicht: Real;
AnzPassagiere: Integer;
END;
\end{lstlisting}

Dieses Beispiel definiert einen neuen Datentyp \textit{Auto}, in dem jeweils die Art des KFZ, das Gewischt und die Anzahl der Passagiere angegeben wird. Nun kann man den Typen \textit{Auto} nutzen um beliebig viele Daten-Entitäten zu erstellen.

\begin{lstlisting}[language=Pascal, caption=Verbund Beispiel Instantiierung \cite{PascalVerbund}]
VAR a : Auto;
BEGIN
a.KFZ := 'B XY 123456';
a.Gewicht := 555.77;
a.AnzPassagiere := 5;
END;
\end{lstlisting}

Manipulationen der Daten innerhalb des Verbunds werden extern vom Datentyp selbst betrachtet, die Syntax der Sprache indiziert somit nur eine Gruppierung von Daten.

Verbunde in Pascal werden homogen genannt, wenn alle Daten innerhalb des Verbunds den selben Typ haben. Sie sind heterogen, wenn es verschiedene Typen, wie im Beispiel, gibt.

%Referenz für Mario: Seite 27 OOPL

\section{Polymorphismus}

\section{Vererbung}\label{sec:Vererbung}

\section{Verwandte Konzepte}

\section{Auswahl des Programmiersprache und Umgebung}


\chapter{Realisierung}\label{sec:Realisierung}
\section{Verwendete Plattformen}
\subsection{Git und Github}
\subsection{Discord}

\section{Zusammenfassung der Anforderungen}

\section{Anforderung 1}
\subsection{Ansatz}
\subsection{Umsetzung}

\section{Anforderung 2}
\subsection{Ansatz}
\subsection{Umsetzung}

\section{Anforderung 3}
\subsection{Ansatz}
\subsection{Umsetzung}

\section{Anforderung 4}
\subsection{Ansatz}
\subsection{Umsetzung}

\chapter{Ergebnisse}

\chapter{Fazit}


\chapter{Ausblick}



%%% Codesnippets

%% Liste
%\textbf{Grundwerte}
%\begin{itemize}\vspace{-1em}
%\setlength{\itemsep}{-1em}
%	\item \textbf{Integrität}: Die Daten oder das System sind nicht von Unbefugten veränderbar. Alle Änderungen müssen nachvollziehbar sein.
%	\item \textbf{Vertraulichkeit}: Die Daten dürfen nur von autorisierten Personen abgerufen und verändert werden. Dies gilt sowohl für gespeicherte Dateien, als auch die Datenübertragung.
%	\item \textbf{Verfügbarkeit}: Die Daten und Systeme müssen erreichbar sein. Systemausfälle müssen vermieden werden. Wenn man eine Verfügbarkeit von "four nines" (99.99\% der Zeit) gewährleisten möchte, dann sind z.B. nur Ausfälle für 52:36 Minuten pro Jahr vertretbar.
%\end{itemize}

%% Bild
%\begin{figure}[H]
%  \centering
%  \includegraphics[scale=0.5]{bilder/Netzplan.JPG}
%  \label{fig:kernabsicherung}       %fig:ID
%  \caption[Beispielhafter Netzplan nach BSI 200-2]{\cite{BSI200_2} Seite 90: Auszug aus dem bereinigten Netzplan der RECPLAST GmbH (Teilausschnitt)}    %Bildunterschrift
%\end{figure}


%% Code
%\begin{lstlisting}[language=csh, caption=Python example]
%public async Task<IActionResult> Archiv()
%    {
%        string UserPersId = User.Identity?.Name ?? "Empty";
%
%        return View(await _context.Dokumente.Where(x => x.PersonalNummer == UserPersId).ToListAsync());
%    }
%//test
%\end{lstlisting}


%%%%%%%%%%%%%%%%%%%%%%%%%%
% Quellen
%%%%%%%%%%%%%%%%%%%%%%%%%

\bibliography{literatur}

\bibliographystyle{hwrbib}
%% \bibliographystyle{alpha} %% tu es nicht, niemals, das ist eklig, nicht einkommentieren

\chapter*{Ehrenwörtliche Erklärung}
\addcontentsline{toc}{chapter}{Ehrenwörtliche Erklärung}

% Keine Kopf- und Fußzeilen ausgeben
\thispagestyle{empty}
% Aber trotzdem ins Inhaltsverzeichnis aufnehmen
%\addcontentsline{toc}{section}{Eidesstattliche Erklärung}

% Hier der offizielle Text der eidesstattlichen Erklärung
Ich erkläre ehrenwörtlich:
\begin{enumerate}
	\item dass ich meine Bachelor-Thesis selbstständig verfasst habe,
	\item dass ich die Übernahme wörtlicher Zitate aus der Literatur sowie die Verwendung der Gedanken anderer Autoren an den entsprechenden Stellen innerhalb der Arbeit gekennzeichnet habe,
	\item dass ich meine Bachelor-Thesis bei keiner anderen Prüfung vorgelegt habe.
\end{enumerate}
Ich bin mir bewusst, dass eine falsche Erklärung rechtliche Folgen haben wird.
% Etwas Abstand für die Unterschrift
\vspace{2cm}

% Hier kommt die Unterschrift drüber
\begin{tabular}{lp{4em}l} 
 \hspace{5cm}   && \hspace{6cm} \\\cline{1-1}\cline{3-3} 
 Ort, Datum     && \studentNameEins \\ [10ex]
 \hspace{5cm}   && \hspace{6cm} \\\cline{1-1}\cline{3-3} 
 Ort, Datum     && \studentNameZwei \\ [10ex]
 \hspace{5cm}   && \hspace{6cm} \\\cline{1-1}\cline{3-3} 
 Ort, Datum     && \studentNameDrei \\ [10ex]
\end{tabular}

\end{document}

