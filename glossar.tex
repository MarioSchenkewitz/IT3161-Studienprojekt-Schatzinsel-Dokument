%%%%%%%%%%%%%%%%%%%%%%%%%%%%%%%%%%%%%%%%%%%%
%%%%% Befehle für Abkürzungen
\newacronym{OOP}{OOP}{Objektorientierte Programmierung}
\newacronym{OOPL}{OOPL}{Objektorientierte Programmiersprachen}

%Abkürzungen mit Glossareintrag

\newacronym{PDF}{PDF}{Portable Document Format\protect\glsadd{glos:PDF}}


%%%%%%%%%%%%%%%%%%%%%%%%%%%%%%%%%%%%%%%%%%%%
%%%%% Befehle für Glossar
\newglossaryentry{glos:IKT}{
name=Dir ZS IKT, 
description={Direktion Zentraler Service Informations- und Kommunikationstechnik, eine Abteilung der Polizei Berlin.}
}

\newglossaryentry{glos:PDF}{
name=Portable Document Format, 
description={Ein nach ISO 32000 standardisiertes Dateiformat um Dokumente, inklusive Textformatierung und Bildern, anzuzeigen.}
}

\newglossaryentry{C}{
name=C\#, 
description={Eine von Microsoft entwickelte Programmiersprache, die vollen Nutzen der .Net Plattform ziehen sollte.}
}

\newglossaryentry{GUI}{
name=Graphical User Interface, 
description={Graphische Benutzer Oberfläche, beschreibt die graphische Darstellung eines Programms auf einem Bildschirm die für Benutzereingaben und für die Ausgabe vom Programm genutzt werden kann. Sie stellt allgemein Informationen und Möglichkeiten für die Interaktion für den Benutzer dar.}
}


%%%%%%%%%%%%%%%%%%%%%%%%%%%%%%%%%%%%%%%%%%%%
%%%%% Befehle für Symbole
%\newglossaryentry{symb:Pi}{
%name=$\pi$,
%description={Die Kreiszahl.},
%sort=symbolpi, type=symbolslist
%}
%\newglossaryentry{symb:Phi}{
%name=$\varphi$,
%description={Ein beliebiger Winkel.},
%sort=symbolphi, type=symbolslist
%}
%\newglossaryentry{symb:Lambda}{
%name=$\lambda$,
%description={Eine beliebige Zahl, mit der der nachfolgende Ausdruck
%multipliziert wird.},
%sort=symbollambda, type=symbolslist
%}