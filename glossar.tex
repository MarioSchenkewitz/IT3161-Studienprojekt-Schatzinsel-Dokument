%%%%%%%%%%%%%%%%%%%%%%%%%%%%%%%%%%%%%%%%%%%%
%%%%% Befehle für Abkürzungen
\newacronym{ISFV}{ISFV}{Informationssicherheits Fachverantwortliche\protect\glsadd{glos:ISFV}}
%Eine Abkürzung mit Glossareintrag

\newacronym{IKT}{Dir ZS IKT}{Direktion Zentraler Server Informations- und Kommunikationstechnik\protect\glsadd{glos:IKT}}
\newacronym{PDF}{PDF}{Portable Document Format\protect\glsadd{glos:PDF}}

\newacronym{ASP.Net}{ASP.Net}{Active Server Pages .NET\protect\glsadd{glos:ASP.Net}}
\newacronym{HTML}{HTML}{HyperText Markup Language\protect\glsadd{glos:HTML}}
\newacronym{CSS}{CSS}{Cascading Style Sheets\protect\glsadd{glos:CSS}}
\newacronym{SQL}{SQL}{Structured Query Language\protect\glsadd{glos:SQL}}
\newacronym{MVC}{MVC}{Model View Controller\protect\glsadd{glos:MVC}}
\newacronym{REST}{REST}{Representational State Transfer\protect\glsadd{glos:REST}}
\newacronym{API}{API}{Application Programming Interface\protect\glsadd{glos:API}}
\newacronym{HTTP}{HTTP}{Hypertext Transfer Protocol\protect\glsadd{glos:HTTP}}
\newacronym{ORM}{ORM}{objektrelationale Abbildung\protect\glsadd{glos:ORM}}

%%%%%%%%%%%%%%%%%%%%%%%%%%%%%%%%%%%%%%%%%%%%
%%%%% Befehle für Glossar
\newglossaryentry{glos:IKT}{
name=Dir ZS IKT, 
description={Direktion Zentraler Service Informations- und Kommunikationstechnik, eine Abteilung der Polizei Berlin.}
}

\newglossaryentry{glos:PDF}{
name=Portable Document Format, 
description={Ein nach ISO 32000 standardisiertes Dateiformat um Dokumente, inklusive Textformatierung und Bildern, anzuzeigen.}
}

\newglossaryentry{C}{
name=C\#, 
description={Eine von Microsoft entwickelte Programmiersprache, die vollen Nutzen der .Net Plattform ziehen sollte.}
}

\newglossaryentry{glos:ASP.Net}{
name=Active Server Pages .NET, 
description={Ein Web Application Framework von Microsoft mit dem sich dynamische Webseite, Webanwendungen und Webservices entwickeln lassen.\\ Ein Web Application Framework dient, wie eine Bibliothek, dem Wiederverwenden von Code, wodurch auf häufig gebrauchte Funktionen zurückgegriffen werden kann.}
}

\newglossaryentry{glos:HTML}{
name=HyperText Markup Language, 
description={Eine textbasierte Auszeichnungssprache, um Dokumente die Texte enthalten mit Hyperlinks, Bildern und anderen Inhalten zu strukturieren.}
}

\newglossaryentry{glos:CSS}{
name=Cascading Style Sheets, 
description={Eine Sprache, um die Präsentation eines Dokumentes zu beschreiben, dass in einer Auszeichnungssprache (markup language), z.B. HTML, geschrieben wurde.}
}

\newglossaryentry{JavaScript}{
name=JavaScript, 
description={JavaScript, auch als JS abgekürzt, ist eine Programmiersprache für Webseiten, die Client-seitige Ausführung von Code erlaubt.}
}

\newglossaryentry{glos:SQL}{
name=Structured Query Language, 
description={Eine Datenbanksprache zur Definition von Datenstrukturen in relationalen Datenbanken.}
}

\newglossaryentry{glos:MVC}{
name=Model View Controller, 
description={Englisch für Modell-Präsentation-Steuerung, ein Muster für die Entwicklung von Software. Es wird zwischen den drei Komponenten des Datenmodells, der Präsentation und der Programmsteuerung unterschieden. Ziel ist ein flexibler Programmentwurf, der spätere Änderungen erleichtert und die Wiederverwendbarkeit der Komponenten ermöglicht.}
}

\newglossaryentry{glos:REST}{
name=Representational State Transfer, 
description={Repräsentationaler Zustands Transfer ist ein Paradigma für die Softwarearchitektur von verteilten Systemen und Webservices. REST ermöglich die Kommunikation von Maschinen miteinander. REST Methoden von HTTP sind unter anderem GET, POST, PUT und DELETE}
}

\newglossaryentry{glos:API}{
name=Application Programming Interface, 
description={Eine Programmierschnittstelle oder auch Anwendungsprogrammierschnittstelle ermöglicht es Softwaresystemen untereinander zu kommunizieren, so können Daten per REST Methoden an eine Schnittstelle gesendet werden, diese Daten werden verarbeitet und eine Antwort zurückgesendet.}
}

\newglossaryentry{glos:HTTP}{
name=Hypertext Transfer Protocol, 
description={Ein zustandsloses Protokoll zur Übertragung von Daten. Es wird hauptsächlich im World Wide Web genutzt, um Webseiten in einem Browser zu laden.}
}

\newglossaryentry{glos:ORM}{
name=objektrelationale Abbildung, 
description={Objekt-relational Mapping, objektrelationale Abbildung, eine Technik mit der in objektorientierten Programmiersprachen geschriebene Anwendungen Objekte in einer relationalen Datenbank ablegen kann.}
}

\newglossaryentry{Framework}{
name=Framework, 
description={Auch Rahmen oder Gerüst, ist ein Ordnungsrahmen mit Bausteinen, der die Wiederverwendung von Programmteilen erlaubt und so das Entwickeln von Anwendungen erleichtert.}
}
%%%%%%%%%%%%%%%%%%%%%%%%%%%%%%%%%%%%%%%%%%%%
%%%%% Befehle für Symbole
%\newglossaryentry{symb:Pi}{
%name=$\pi$,
%description={Die Kreiszahl.},
%sort=symbolpi, type=symbolslist
%}
%\newglossaryentry{symb:Phi}{
%name=$\varphi$,
%description={Ein beliebiger Winkel.},
%sort=symbolphi, type=symbolslist
%}
%\newglossaryentry{symb:Lambda}{
%name=$\lambda$,
%description={Eine beliebige Zahl, mit der der nachfolgende Ausdruck
%multipliziert wird.},
%sort=symbollambda, type=symbolslist
%}